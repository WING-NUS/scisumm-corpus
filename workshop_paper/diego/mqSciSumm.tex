%
% File acl-hlt2011.tex
%
% Contact: gdzhou@suda.edu.cn
%%
%% Based on the style files for ACL2008 by Joakim Nivre and Noah Smith
%% and that of ACL2010 by Jing-Shin Chang and Philipp Koehn


\documentclass[11pt]{article}
\usepackage{acl-hlt2011}
\usepackage{times}
\usepackage{latexsym}
\usepackage{amsmath}
\usepackage{multirow}
\usepackage{url}
\DeclareMathOperator*{\argmax}{arg\,max}
\setlength\titlebox{6.5cm}    % Expanding the titlebox

\title{Macquarie University at TAC 2014 SciSumm}

\author{Diego Moll\'a \hfill Christopher jones\\
  Macquarie University\\
  Sydney, Australia\\
  {\tt diego.molla-aliod@mq.edu.au}\\
  {\tt christopher.jones4@students.mq.edu.au}}

\date{}

\begin{document}
\maketitle
\begin{abstract}
  For the SciSumm track, we used the same settings as in our
  participation in the TAC 2014 BiomedSumm track. In this report we
  explain our experiments for the SciSumm track. We tried the same
  runs as for our participation in the TAC 2014 BiomedSumm track,
  except for runs that use UMLS since the domain of the documents in
  the SciSumm track was not biomedical. We also tried a variant of
  task 2 of the BiomedSumm track, where we aim to recreate the
  document abstracts by incorporating information from the citing
  papers. In general, the impact of the different methods used in the
  runs was the same as for the BiomedSumm track, but the absolute
  values of the results was lower.
\end{abstract}

\section{Introduction}

We wanted to test whether the results obtained in our runs in the TAC
2014 BiomedSumm track were also applicable to the domain of
SciSumm. In particular, we replicated the runs of task1a, except for a
run that used UMLS. In addition, we tried the same runs of task2 for a
variant of task 2 where the goal is to re-create the abstract section
of the reference paper. We observed that the impact of information
from the citances is positive for biomedical publications
\cite{Molla:ALTA2014}, and we wanted to know whether this is also true
for the Computational Linguistics domain.

\section{The MQ System - Finding the Best Fit to a Citance}

Given the text of a citance, our system ranks the sentences of the
reference paper according to its similarity to the citance. In all
cases we modelled each sentence and citance as a vector, and used
cosine similarity. We experimented with different forms of
representing the information in the vectors, and different forms of
using the similarity scores to perform the final sentence ranking.

\subsection{Using \emph{tf.idf}}\label{sec:tfidf}

In our simplest approach, we computed the \emph{tf.idf} of all
lowercased words, without removing stop words. We computed separate
\emph{tf.idf} statistics for each reference paper. In particular, for
each reference paper we computed the \emph{idf} component using the
set of sentences in the paper and the citance text of all citing
papers.

\subsection{Adding texts of the same topic}\label{sec:topics}

Since the amount of text used to compute the \emph{tf.idf} in
Section~\ref{sec:tfidf} was relatively little, we extended the
information used to compute the \emph{idf} component by adding the
complete text of all citing papers. Given that the citing papers are
presumably of the same topic as the reference paper, by adding this
text we hope to include complementary information that can be useful
for computing the \emph{idf} information.

\subsection{Adding context}\label{sec:context}

Further to the approach in Section~\ref{sec:topics}, we extended the
information of each sentence in the reference paper by including the
neighouring sentences within a context window of 20 sentences centered
in the target sentence.

\subsection{Re-ranking using MMR}\label{sec:mmr}

The last experiment used Maximal Marginal Relevance (MMR)
\cite{Carbonell:1998} to rank the sentences. All sentences were
represented as \emph{tf.idf} vectors of extended information as
described in Section~\ref{sec:context}. Then, the final score of a
sentence was the combination of the similarity with the citance and
similarity of the other sentences of the summary according to the
formula shown in Figure~\ref{fig:mmr}. We chose a value of
$\lambda=0.97$.

\begin{figure*}
$$
\hbox{MMR} = \arg\max_{D_i\in R\setminus S}\left[\lambda(\hbox{sim}(D_i,Q)) -
(1-\lambda) \max_{D_j\in S} \hbox{sim}(D_i,D_j)\right]
$$  
\begin{quote}
Where:
\begin{itemize}
\item $Q$ is the citance text.
\item $R$ is the set of sentences in the document.
\item $S$ is the set of sentences that haven been chosen in the
  summary so far.  
\end{itemize}
\end{quote}
  \caption{Maximal Marginal Relevance (MMR)}
  \label{fig:mmr}
\end{figure*}

\subsection{Evaluation and Results}

Our evaluation is based on ROUGE \cite{Lin:2004}. ROUGE is a popular
evaluation method for summarisation systems that compares the text
output of the system against a set of target summaries. In our case,
the output is the set of selected sentences, and the target summaries
are the sentences given by the annotators. Since ROUGE uses the actual
contents words, and not the offset information, we expect that this
metric will give non-zero results for cases when the system chooses a
sentence that is similar to, but not exactly, the one chosen by the
annotator.

Since all of the approaches were unsupervised, we used all the data
without having to perform cross-validation experiments.

Table~\ref{tab:task1a} summarises the results of our experiments with
both the SciSumm data and the BiomedSumm data. In all results the
systems were designed to return 3 sentences, as specified in the
shared task. We also ignored all short sentences (under 50 characters)
to avoid including headings or mistakes made by the sentence
segmentation algorithm.

\begin{table*}
  \centering
  \begin{tabular}{l|rrrc|rrrc}
& \multicolumn{4}{|c|}{SciSumm} & \multicolumn{4}{c}{BiomedSumm}\\
Run & R & P & F1 & CI & R & P & F1 & CI\\
\hline
    \emph{tf.idf} & 0.316 & 0.198 & 0.211 & 0.185--0.240 & 0.273 &
    0.326 & 0.279 & 0.265--0.293\\
topics & 0.324 & 0.201 & 0.217 & 0.191--0.245 & 0.288 & 0.357 & 0.300
& 0.285--0.316\\
context & 0.339 & 0.214 & 0.225 & 0.197--0.255 & 0.291 & 0.372 & 0.308
& 0.293--0.323\\
MMR & 0.335 & 0.212 & 0.223 & 0.195--0.251 &  0.290 & 0.375 & 0.308 & 0.293--0.323\\ 
  \end{tabular}
  \caption{ROUGE-L results of our runs for task 1a}
  \label{tab:task1a}
\end{table*}

We observe a similar improvement of results in both domains, with the
exception that MMR does not improve over the run that uses
\emph{tf.idf} over context in SciSumm, whereas there is an improvement
in BiomedSumm. The absolute values are better in the BiomedSumm data,
and looking at the confidence intervals we can presume that the
difference between the best and the worst run is statistically
significant in the BiomedSumm data. The results in the SciSumm data
are poorer in general and we cannot find any values that are
statistically significant. But given that the amount of data in
SciSumm is smaller, and given that the improvement mirrors that of the
BiomedSumm data, it seems that, in general, as we add more information
to the models that compute \emph{tf.idf}, the results improve. So we
can presume that approaches that gather related information to be
added for computing the vector models will produce even better
results. The results with MMR appears to be contradictory across the
two domains but the difference is so small that it might not be
statistically significant even when we add more evaluation data.

Table~\ref{tab:task1av2} shows the ROUGE-L F1 scores of each
individual reference document from the SciSumm dataset, for comparison
with runs by other teams.

\begin{table*}
  \centering
  \begin{tabular}{l|rrrr}
Paper ID & \emph{tf.idf} & topics & context & MMR\\
\hline
C90-2039\_TRAIN & 0.235 & 0.229 & 0.235 & 0.228\\
C94-2154\_TRAIN & 0.298 & 0.307 & 0.288 & 0.288\\
E03-1020\_TRAIN & 0.274 & 0.240 & 0.239 & 0.243\\
H05-1115\_TRAIN & 0.209 & 0.271 & 0.350 & 0.311\\
H89-2014\_TRAIN & 0.302 & 0.315 & 0.332 & 0.333\\
J00-3003\_TRAIN & 0.209 & 0.206 & 0.196 & 0.199\\
J98-2005\_TRAIN & 0.105 & 0.113 & 0.101 & 0.108\\
N01-1011\_TRAIN & 0.214 & 0.228 & 0.221 & 0.219\\
P98-1081\_TRAIN & 0.180 & 0.192 & 0.200 & 0.201\\
X96-1048\_TRAIN & 0.238 & 0.235 & 0.248 & 0.247\\
\hline
Micro-average & 0.211 & 0.217 & 0.225 & 0.223\\
Macro-average & 0.226 & 0.234 & 0.241 & 0.238

  \end{tabular}
  \caption{ROUGE-L F1 results for individual SciSumm reference papers for task
  1a}
  \label{tab:task1av2}
\end{table*}

\section{Building the Final Summary}

We wanted to test whether information from the citances are useful for
building an extractive summary, as is the case with the BiomedSumm
data \cite{Molla:ALTA2014}. We therefore implemented extractive
summarisation systems with and without information from the citances.

The summarisers without information from the citances scored each
sentence as the sum of the \emph{tf.idf} values of the sentence
elements. We tried the \emph{tf.idf} approach described in
Section~ref{sec:tfidf}.

The summarisers with information from the citances scored each
candidate sentence $i$ on the basis of rank($i$,$c$) obtained in task
1a, which has values between 0 (first sentence) and $n$ (last
sentence) and represents the rank of sentence $i$ in citance $c$:

$$
\hbox{score}(i) = \sum_{c\in\hbox{citances}}1-\frac{\hbox{rank}(i,c)}{n}
$$

The summaries were evaluated using ROUGE-L, where the model summaries
are the abstract section of the corresponding papers. Since paper
X96-1048 of the SciSumm data did not have an abstract section, it was
removed for this experiment. One of the 20 documents from
the BiomedSumm was also removed for the same
reason. Table~\ref{tab:task2} shows the evaluation results.

\begin{table*}
  \centering
  \begin{tabular}{l|rrrc|rrrc}
& \multicolumn{4}{|c|}{SciSumm} & \multicolumn{4}{c}{BiomedSumm}\\
Run & R & P & F1 & CI & R & P & F1 & CI\\
\hline
\emph{tf.idf} &  0.379 & 0.157 & 0.214 & 0.168--0.264 & 0.293 & 0.192
& 0.227 & 0.190--0.261\\
\hline
task1a \emph{tfidf} & 0.396 & 0.203 & 0.259 & 0.207--0.307 & 0.425 &
0.266 & 0.322 & 0.294--0.355\\
task1a MMR & 0.406 & 0.199 & 0.260 & 0.209--0.306 & 0.480 & 0.302 & 0.364 & 0.332--0.397\\
  \end{tabular}
  \caption{ROUGE-L results of our runs for task 2}
  \label{tab:task2}
\end{table*}

Again, we observe a similar pattern in both domains, and again the
absolute values of the results are higher in the BiomedSumm task. The
versions that use the data from task1a are better than the versions
that do not use information from task1a. The difference is
statistically significant with the BiomedSumm data, and it is near the
edge of statistically significance with the SciSumm data.

Table~\ref{tab:task2v2} shows the breakout of ROUGE-L F1 scores per
document, for comparison with the runs by other
teams.\footnote{Table~\ref{tab:task2v2} has a row ``Average'' instead
  of two rows for micro- and macro-average because both micro- and
  macro-average have the same values.}

\begin{table*}
  \centering
  \begin{tabular}{l|r|rrr}
Paper ID & \emph{tf.idf} & task1a \emph{tf.idf}  & task1a MMR\\
\hline
C90-2039\_TRAIN & 0.347 & 0.315 & 0.293\\
C94-2154\_TRAIN & 0.095 & 0.123 & 0.120\\
E03-1020\_TRAIN & 0.189 & 0.189 & 0.196\\
H05-1115\_TRAIN & 0.134 & 0.306 & 0.321\\
H89-2014\_TRAIN & 0.294 & 0.319 & 0.320\\
J00-3003\_TRAIN & 0.221 & 0.382 & 0.367\\
J98-2005\_TRAIN & 0.221 & 0.216 & 0.233\\
N01-1011\_TRAIN & 0.187 & 0.268 & 0.284\\
P98-1081\_TRAIN & 0.241 & 0.210 & 0.206\\
\hline
Average & 0.214 & 0.259 & 0.260
  \end{tabular}
  \caption{ROUGE-L F1 results for individual SciSumm reference papers for task
  2}
  \label{tab:task2v2}
\end{table*}

\section{Conclusions}

The results of the experiments reported in this paper suggest that
information from related papers may be useful to find the sentences of the
reference paper that best match the citances. 

Our experiments also suggest that information from the citances may be
useful for building an extractive summary. This conclusion is
compatible with prior research that suggest that, in general,
information from citing papers may be useful for building summaries,
as was stated in the original goals of the BiomedSumm and SciSumm
shared tasks.


\section*{Acknowledgments}

This research was made possible thanks to a summer internship granted
to Christopher Jones by the Department of Computing, Macquarie
University.

\bibliographystyle{acl}
\bibliography{mqSciSumm}

% \begin{thebibliography}{}

% \bibitem[\protect\citename{Aho and Ullman}1972]{Aho:72}
% Alfred~V. Aho and Jeffrey~D. Ullman.
% \newblock 1972.
% \newblock {\em The Theory of Parsing, Translation and Compiling}, volume~1.
% \newblock Prentice-{Hall}, Englewood Cliffs, NJ.

% \bibitem[\protect\citename{{American Psychological Association}}1983]{APA:83}
% {American Psychological Association}.
% \newblock 1983.
% \newblock {\em Publications Manual}.
% \newblock American Psychological Association, Washington, DC.

% \bibitem[\protect\citename{{Association for Computing Machinery}}1983]{ACM:83}
% {Association for Computing Machinery}.
% \newblock 1983.
% \newblock {\em Computing Reviews}, 24(11):503--512.

% \bibitem[\protect\citename{Chandra \bgroup et al.\egroup }1981]{Chandra:81}
% Ashok~K. Chandra, Dexter~C. Kozen, and Larry~J. Stockmeyer.
% \newblock 1981.
% \newblock Alternation.
% \newblock {\em Journal of the Association for Computing Machinery},
%   28(1):114--133.

% \bibitem[\protect\citename{Gusfield}1997]{Gusfield:97}
% Dan Gusfield.
% \newblock 1997.
% \newblock {\em Algorithms on Strings, Trees and Sequences}.
% \newblock Cambridge University Press, Cambridge, UK.

% \end{thebibliography}

\end{document}
